\chapter{Introduction}
\label{chap:intro}


%%%%%%%%%%%%%%%%%%%%%%%%%%%%%%%%%%%%%%%%%%%%%%%%%%%%%%%%%%%%%%%%%%%%%%%%%%%%%%%%%%%%%%%%%%%%%
%%%%%%%%%%%%%%%%%%%%%%%%%%%%%%%%%%%%%%%%%%%%%%%%%%%%%%%%%%%%%%%%%%%%%%%%%%%%%%%%%%%%%%%%%%%%%
\section{Motivation of the Deliverable}
\label{sec:motiv}

The fashion domain involves multiple industry players e.g., fashion designers, production chains, retailers, marketing companies, value-added services, \ac{SM} outlets etc. Each player produces important pieces of information that can be used to provide a better end-user experience, improve individual processes and drive more profit. Fashion data has recently gained a lot of attention in various fashion applications e.g., clothing recommendation, clothing recognition, fashion trend prediction, etc. Yet, the relationship between datasets emanating from retailers and social media is rarely explored. One major impediment is the distribution of the data sources containing relevant products information. Once we uncover these datasources, we are still faced with the challenge of integrating them into a federated data repository. For instance, two online retailers might deal with common brands yielding redundant items in their datasets, thus, the need of eliminating redundant data. In the context of the FashionBrain project, the datasets provided by our partners will be used in order to i) automatically process text and image data and extract key features such as fashion items, ii) capture customers’ preferences, iii) build a fashion-based recommendation engine and iv) predict new fashion trends as they emerge on \ac{SM} and the blogosphere. 


%%%%%%%%%%%%%%%%%%%%%%%%%%%%%%%%%%%%%%%%%%%%%%%%%%%%%%%%%%%%%%%%%%%%%%%%%%%%%%%%%%%%%%%%%%%%%
%%%%%%%%%%%%%%%%%%%%%%%%%%%%%%%%%%%%%%%%%%%%%%%%%%%%%%%%%%%%%%%%%%%%%%%%%%%%%%%%%%%%%%%%%%%%%